\documentclass[a4paper,10pt]{article}

\usepackage[english]{babel}
\usepackage[utf8]{inputenc}
\usepackage{graphicx}
\usepackage{amsmath,amssymb}
\usepackage{hyperref, url}
\usepackage{listings}
\usepackage[margin=1in]{geometry}
\usepackage{float}


\begin{document}
\title{Muistiinpanoja teoriasta yms.}
\author{}
\date{}
\maketitle

Hyödyllisiä linkkejä: \url{http://arxiv.org/pdf/1007.1245.pdf} ja 
\url{http://fab.cba.mit.edu/classes/864.14/students/Skuhersky_Michael/explosion/index.html}.

Seurattavia suureita hiukkasten sijainnit ja nopeudet. Näistä lasketaan kullakin aika-askeleella
tiheydet ja kiihtyvyydet. Kiihtyvyyksiä varten tarvitaan lauseke paineelle. Tämä voidaan laskea
yleensä tiheyden avulla, mutta usein tiheyden lausekkeessa tunnutaan käyttävän myös räjähdysrintaman
etenemisnopeutta kuvaavaa termiä.

Tiheyden laskeminen hiukkasen $j$ kohdalla:
\begin{equation}
 \rho_j = \sum_{i \neq j} m_i W_{ji}.
\end{equation}
Kiihtyvyyden laskeminen:
\begin{equation}
 \mathbf{a}_j = - \sum_{i \neq j} m_i \left( \frac{P_j}{\rho_j^2} + \frac{P_i}{\rho_i^2} \right) \nabla_j W_{ji}.
\end{equation}
Tähän perään voisi lisäillä laskentametodeja paineelle hiukkasen i kohdalla. Yksinkertaisin formaatti olisi
tiheyteen verrannollinen paine (ideaalikaasu).

Leap-frog algoritmi: lasketaan ensin nopeus ja paikka puolen aika-askelen päässä.
\begin{eqnarray}
 \mathbf{v}_{i+1/2} = \mathbf{v}_i + \frac{\Delta t}{2} \mathbf{a}_i \\
 \mathbf{x}_{i+1/2} = \mathbf{x}_i + \frac{\Delta t}{2} \mathbf{v}_i
\end{eqnarray}
Lasketaan näistä tarpeelliset suureet (kuten tiheys) puolen aika-askelen päässä. Nyt
\begin{eqnarray}
 \mathbf{v}_{i+1} = \mathbf{v}_i + \Delta t \mathbf{a}_{i+1/2} \\
 \mathbf{x}_{i+1} = \mathbf{x}_i + \frac{\Delta t}{2} ( \mathbf{v}_i + \mathbf{v_{i+1}} )
\end{eqnarray}
Huomionarvoista erot käsittelyssä matkalla puolen aika-askelen päähän ja kokonaisen aika-askelen päähän,
muutenhan missään ei olisi mitään järkeä.


\end{document}