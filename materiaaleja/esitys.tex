\documentclass{beamer}
\usepackage[latin1]{inputenc}
\usefonttheme{serif} 
\usefonttheme{structuresmallcapsserif} 

\usetheme{Luebeck}

\usecolortheme[named=magenta]{structure}
\beamertemplatenavigationsymbolsempty
\setbeamertemplate{bibliography item}[text]
\title[Explosions]{Explosions}
\subtitle{Computational Physics Presentation}
\author{Ville, Aapo, Hannu}


\date{April 28, 2015}



\begin{document}

\begin{frame}
\titlepage
\end{frame}

\begin{frame}{Introduction}
\begin{itemize}

\item A nice group project? $\rightarrow$ Do something completely different than during the course
\item Idea: simulate explosions - a non-trivial and fairly interesting subject
\item The actual implementation studies effectively an explosion of gas particles in a vacuum
\item Unrealistic? No - very similar to the intuitive conception of explosions
\item In an explosion the density of the gas that explodes is much higher than that of the surroundings
\item Only when the explosive gas has expanded enough, the surrounding density becomes important - and this regime is not of interest here

\end{itemize}
\end{frame}


\begin{frame}{Physics of an explosion}
\begin{itemize}

\item High density, pressure or temperature with respect to surroundings $\rightarrow$ explosion
\item Start with a high-density gas cloud $\rightarrow$ the physics of the system will take care of the rest
\item Required: fluid dynamics for compressible systems
\item A shock wave should be produced to the progressing explosion front
\item Handling of dynamic length scales: begin with a compressed system and end with an expanded system
\item Only the dynamical properties of the system are studied, so temperature does not require explicit handling

\end{itemize}
\end{frame}

\begin{frame}{Fluid simulation}
\begin{itemize}

\item Simulating the behaviour of a fluid is highly non-trivial
\item Handling a quickly expanding fluid is a non-trivial case within the group of fluid simulations
\item A solution from astrophysics: smoothed particle hydrodynamics
\item Astrophysics is familiar with changing length scales and vacuum boundary conditions - this fits well a simulation of explosions
\item The basic idea is to model the fluid with a group of discrete particles
\item Fluid physics are obtained by applying a smoothing kernel function to the particles

\end{itemize}
\end{frame}

\begin{frame}{SPH basics}
\begin{itemize}

\item Dirac delta function $\delta (\mathbf{x})$ can be approximated using  a kernel function $W(\mathbf{x},h)$ (for instance Gaussian)
\item A function can be represented using delta functions $\rightarrow$
\begin{equation}
 f(\mathbf{x}) = \int_V f(\mathbf{x}') \delta (\mathbf{x}-\mathbf{x}') d\mathbf{x}' \approx \int_V \frac{f(\mathbf{x}')}{\rho(\mathbf{x}')} W(\mathbf{x}-\mathbf{x}',h) \rho(\mathbf{x}') d\mathbf{x}' 
\end{equation}
\item Approximate integral over the density with a sum over point masses:
\begin{equation}
 f(\mathbf{x}) \approx \sum_i m_i \frac{f(\mathbf{x}_i)}{\rho_i} W( \mathbf{x} - \mathbf{x_i}, h )
\end{equation}
\item Now for instance the density can be calculated:
\begin{equation}
\rho(\mathbf{x}) \approx \sum_i m_i W( \mathbf{x} - \mathbf{x_i}, h )
\end{equation}

\end{itemize}
\end{frame}


\begin{frame}{SPH kernel functions}
\begin{itemize}

\item sus

\end{itemize}
\end{frame}

\end{document}